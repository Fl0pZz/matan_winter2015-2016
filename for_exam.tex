\documentclass{report}
\usepackage[utf8]{inputenc}
\usepackage[russian]{babel} 

\usepackage{amssymb}
\usepackage{amsmath}
\usepackage[normalem]{ulem}


\usepackage{geometry} 
\geometry{letterpaper}
\usepackage{graphicx}
\usepackage{mathtools}

\ULdepth = 0.16em

\begin{document}
\section{Вопрос 1.}

Числовой ряд $\sum\limits_{k=1}^\infty a_k$ называется \textbf{\uline{сходящимся}}, если существует конечный предел последовательности частичных сумм $S = \lim\limits_{n \to \infty} S_n$. Если предел последовательности частичных сумм числового ряда не существует или бесконечен, то ряд $\sum\limits_{k=1}^\infty a_k$ называется \textbf{\uline{расходящимся}}.\\\\

\textbf{\uline{Суммой сходящегося числового ряда}} $\sum\limits_{k=1}^\infty a_k$ называется предел последовательности его частичных сумм, то есть, $\sum\limits_{k=1}^\infty a_k = \lim\limits_{n \to \infty} S_n = S$.\\\\

\textbf{\uline{Арифметические свойства}}.\\
Пусть даны числовые последовательности $\sum\limits_{k=1}^\infty x_k \to a$ и $\sum\limits_{k=1}^\infty y_k \to b$, тогда верны следующие свойства:
\begin{enumerate}
    \item $\lim\limits_{n \to \infty} (c_1 \cdot x_n \pm c_2 \cdot y_n) = c_1 \cdot \lim\limits_{n \to \infty} x_n \pm c_2 \cdot \lim\limits_{n \to \infty} y_n = c_1 \cdot a \pm c_2 \cdot b$
    
    \item $\lim\limits_{n \to \infty} ((c_1 \cdot x_n) \cdot (c_2 \cdot y_n)) = c_1 \cdot \lim\limits_{n \to \infty} x_n \cdot c_2 \cdot \lim\limits_{n \to \infty} y_n = c_1 \cdot a \cdot c_2 \cdot b$
    
    \item $\lim\limits_{n \to \infty} \frac{c_1 \cdot x_n}{c_2 \cdot y_n} = \frac{c_1 \cdot \lim\limits_{n \to \infty} x_n}{c_2 \cdot \lim\limits_{n \to \infty} y_n} = \frac{c_1 \cdot a}{c_2 \cdot b}, b \ne 0, y_n \ne 0, c_2 \ne 0$
\end{enumerate}

\textbf{\uline{Свойства остатков}}.\\
Если ряд $\sum\limits_{k=1}^\infty a_k$ сходится, то сходится и любой его остаток. Если сходится какой-нибудь остаток ряда, то сходится и сам ряд.\\\\

\textbf{\uline{Свойства группировки}}.\\
Если ряд $A = \sum\limits_{k=1}^\infty a_k$ сходится, то ряд $B = \sum\limits_{j=1}^\infty b_j$ полученный путем группировки членов ряда $A$ без изменения порядка их расположения, также сходится и его сумма равна сумме ряда $A$.\\

\textbf{\uline{Задача}}\\
Найдите n-ю частичную сумму и сумму ряда $\sum\limits_{n = 1}^\infty \ln{(1 - n^{-2})}$.\\

\underline{Решение:}\\
С помощью необходимого признака сходимости, проверим, не расходится ли ряд:
$$\lim\limits_{n \to \infty} \ln{(1 - n^{-2})} = \ln{\lim\limits_{n \to \infty} (1 - n^{-2})} = \ln{1} = 0$$
Рассмотрим, что сокращается при суммировании $a_{n-1}$, $a_n$ и $a_{n+1}$:
\begin{multline*}
    a_{n-1} + a_{n} + a_{n+1} = \ln{(1 - (n-1)^{-2})} + \ln{(1 - n^{-2})} + \ln{(1 - (n+1)^{-2})} = \\ = \ln{\frac{(n-1)^2 - 1}{(n-1)^2}} + \ln{\frac{n^2 - 1}{n^2}} + \ln{\frac{(n+1)^2 - 1}{(n+1)^2}} = \ln{\frac{(n-1)-1}{n-1} \cdot \frac{(n-1)+1}{n-1}} + \\ + \ln{\frac{n-1}{n} \cdot \frac{n+1}{n}} + \ln{\frac{(n+1)-1}{n+1} \cdot \frac{(n+1)+1}{n+1}} = (\ln{(n)} - \ln{(n-1) + \ln{(n-2)} - \ln{(n-1)}}) + \\ + (\ln{(n + 1)} - \ln{(n) + \ln{(n-1)} - \ln{(n)}}) + (\ln{(n + 2)} - \ln{(n + 1) + \ln{(n)} - \ln{(n + 1)}}) = \\ = \ln{(n-2)} - \ln{(n-1)} + \ln{(n)} - \ln{(n + 1)}
\end{multline*}
Тогда $$S_n = \sum\limits_{k = 2}^n \ln{(1 - n^{-2})} = \ln{(2 - 1)} - \ln{2} + \ln{(n + 1)} - \ln{(n)} = \ln{\frac{n+1}{2n}}$$
Откуда следует, что $$\lim\limits_{n \to \infty} \sum\limits_{k = 2}^n \ln{(1 - n^{-2})} = \lim\limits_{n \to \infty} \ln{\frac{n+1}{2n}} = \ln(\frac{1}{2}) = -\ln{2}$$

\newpage
\section{Вопрос 3.}
\textbf{\uline{Критерий Коши}}.\\
Ряд $\sum\limits_{n = 1}^\infty z_n$ сходится тогда и только тогда, когда для него выполняется условие Коши: для каждого $\varepsilon > 0$ существует номер $N_{\varepsilon}$ такой, что для любого $n \ge N_{\varepsilon}$ и для любого $p \in N$ справедливо равенство:
$$|z_{n+1} + Z_{n+2} + \cdots +Z_{n+p}| < \varepsilon$$
Или другими словами:
$$\forall \varepsilon > 0 \exists N_{\varepsilon} \forall n \ge N_{\varepsilon} \forall p \in N: |z_{n+1} + Z_{n+2} + \cdots +Z_{n+p}| < \varepsilon$$

\textbf{\uline{Доказательство расходимости гармонического ряда}}.\\
Применим доказательство от противного , предположим, что гармонический ряд сводится к сумме $S$:
$$\sum\limits_{k = 1}^\infty \frac{1}{k} = 1 + \frac{1}{2} + \frac{1}{3} + \frac{1}{4} + \cdots = S$$
Гармонический ряд можно представить в виде суммы 2х рядов:
$$S = (1 + \frac{1}{3} + \frac{1}{5} + \frac{1}{7} + \cdots) + (\frac{1}{2} + \frac{1}{4} + \frac{1}{6} + \frac{1}{8} + \cdots)$$
$$S = (1 + \frac{1}{3} + \frac{1}{5} + \frac{1}{7} + \cdots) + \frac{1}{2} \cdot (1 + \frac{1}{2} + \frac{1}{3} + \frac{1}{4} + \cdots)$$
$$S = (1 + \frac{1}{3} + \frac{1}{5} + \frac{1}{7} + \cdots) + \frac{1}{2}S$$
$$\frac{1}{2}S = 1 + \frac{1}{3} + \frac{1}{5} + \frac{1}{7} + \cdots$$
$$\frac{1}{2} + \frac{1}{4} + \frac{1}{6} + \frac{1}{8} + \cdots = 1 + \frac{1}{3} + \frac{1}{5} + \frac{1}{7} + \cdots$$
$$(1 - \frac{1}{2}) + (\frac{1}{3} - \frac{1}{4}) + \cdots \ne 0$$
Отсюда получаем, что наше предположение не верно, а значит гармонический ряд расходится.

\newpage
\section{Вопрос 5.}
Доказательства сами придумайте, я их в уши еб.\\

\textbf{\uline{Признак Коши}}\\
Пусть $\sum\limits_{k = 1}^\infty a_n$ неотрицательна, тогда $\lim\limits_{n \to \infty} \sqrt[n]{a_n} = \lambda$. При $\lambda > 1$ ряд расходится, при $\lambda < 1$ ряд сходится.\\

\textbf{\uline{Интрегральный признак сходимости}}\\
Пусть $f(x)$ монотона на $[1, \infty]$. $\sum\limits_{n = 1}^\infty f(x)$ сходится $\Leftrightarrow$ сходится $\int\limits_{1}^\infty f(x) dx$.\\

\textbf{\uline{Признак Даламбера}}\\
Пусть $\sum\limits_{k = 1}^\infty a_n$ неотрицательна, $\lim\limits_{n \to \infty} \frac{a_{n+1}}{a_n} = \lambda$. При $\lambda > 1$ ряд расходится, при $\lambda < 1$ ряд сходится.\\

\textbf{\uline{Задача}}\\
Исследуйте на сходимость ряд $\sum\limits_{n = 1}^\infty \frac{(2n)!!}{n!} \arcsin{(3^{-n})}$

\textbf{\uline{Решение}}\\
Так как $3^{-n}$, при $n \to \infty$ стремится к $0$, то $\arcsin{x} \sim x$.
Еще можно заметить, что $\frac{(2n)!!}{n!} = 2^n$.
Воспользуемся признаком сходимости Коши:
$$\lim\limits_{n \to \infty} \sqrt[n]{\frac{(2n)!!}{n!} \arcsin{(3^{-n})}} = \lim\limits_{n \to \infty} \sqrt[n]{2^n \cdot 3^{-n}} = \lim\limits_{n \to \infty} \sqrt[n]{\left(\frac{2}{3}\right)^n} = \frac{2}{3} < 1$$
Откуда понятно, что ряд сходится.
\newpage

\section{Вопрос 6.}
Ряд называется \textbf{\uline{знакочередующимся}}, если его члены попеременно принимают значения противоположных знаков, т. е.:
$$\sum\limits_{n = 1}^\infty b_n = \sum\limits_{n = 1}^\infty (-1)^{n}a_n, a_n > 0$$
\textbf{\uline{Признак Лейбница}}\\
Если для знакочередующегося ряда $\sum\limits_{n = 1}^\infty b_n = \sum\limits_{n = 1}^\infty (-1)^{n}a_n, a_n > 0$ выполняются следующие условия:
\begin{enumerate}
    \item $a_{n+1} < a_n$ (монотонное убывание $\{ a_n\}$)
    \item $\lim\limits_{n \to \infty} a_n = 0$
\end{enumerate}
Тогда этот ряд сходится.\\\\
\textbf{\uline{Задача}}\\
Исследуйте на сходимость ряд $\sum\limits_{n = 1}^\infty (-1)^{n} \left(1 - \cos{\frac{\pi}{\sqrt{n}}}\right)$.\\
\textbf{\uline{Решение}}\\
Воспользуемся признаком Лейбница:\\
\begin{enumerate}
    \item $1 - \cos{\frac{\pi}{\sqrt{n + 1}}} < 1 - \cos{\frac{\pi}{\sqrt{n}}}$, т.к $\cos{\frac{\pi}{\sqrt{n+1}}} > \cos{\frac{\pi}{\sqrt{n}}}$
    \item $\lim\limits_{n \to \infty} \left( 1 - \cos{\frac{\pi}{\sqrt{n}}} \right) = 1 - 1 = 0$
\end{enumerate}
Откуда следует, что ряд сходится.
\newpage

\section{Вопрос 7.}
Ряд $\sum\limits_{n = 1}^\infty a_n$ является абсолютно сходящимся, если сходитcя ряд из его модулей: $\sum\limits_{n \to \infty}^\infty |a_n|$.
\textbf{\uline{Свойство 1}}: Абсолютно сходяйщися ряд сходится, т.е из сходимости ряда $\sum\limits_{n \to \infty}^\infty |a_n|$ следует сходимость ряда $\sum\limits_{n = 1}^\infty a_n$, причем $|S| \le \sigma$,  где $S = \sum\limits_{n = 1}^\infty a_n$, а $\sigma = \sum\limits_{n \to \infty}^\infty |a_n|$.\\\\
\textbf{\uline{Свойство 2}}: Если ряды  $\sum\limits_{n = 1}^\infty a_n$ и $\sum\limits_{n = 1}^\infty b_n$ абсолютно сходятся, то при любях $\alpha$ и $\beta$ ряд\\ $\sum\limits_{n = 1}^\infty (\alpha a_n + \beta b_n)$ также абсолютно сходится.\\\\
\textbf{\uline{Свойство 3}}: Если ряд $\sum\limits_{n = 1}^\infty a_n$ абсолютно сходится, то ряд, составленый из тех же членов, но взятых в другом порядке, также абсолютно сходится, и его сумма равна сумме исходного ряда.\\\\
\textbf{\uline{Свойство 4}}: Если ряды  $\sum\limits_{n = 1}^\infty a_n$ и $\sum\limits_{n = 1}^\infty b_n$ абсолютно сходятся, то ряд, составленный из всевозможных попарных произведений $a_i \cdot b_i$ членов этих рядов, расположенных в любом порядке, также абсолютно сходится, а его сумма равна $S\sigma$, где $S = \sum\limits_{n = 1}^\infty a_n$, $\sigma = \sum\limits_{n = 1}^\infty b_n$.\\\\
\textbf{\uline{Задача}}\\
Исследуйте на абсолютную и условную сходимости ряд $\sum\limits_{n = 3}^\infty \frac{\ln^2{(\ln{n})}}{n \ln{n}} \cos{\pi n}$.\\
\textbf{\uline{Решение}}\\
Заметим, что $\sum\limits_{n = 3}^\infty \frac{\ln^2{(\ln{n})}}{n \ln{n}} \cos{\pi n} = \sum\limits_{n = 3}^\infty \frac{\ln^2{(\ln{n})}}{n \ln{n}} \cdot (-1)^n$.
Предположим, что ряд абсолютно сходится. Тогда рассмотрим сходимость ряда:
$$\sum\limits_{n = 3}^\infty \left | \frac{\ln^2{(\ln{n})}}{n \ln{n}} \cdot (-1)^n \right | = \sum\limits_{n = 3}^\infty \frac{\ln^2{(\ln{n})}}{n \ln{n}}$$
Воспользуемся интегральным признаком ($t = \ln{n}$, $k = \ln{t}$):
$$\int\limits_3^\infty \frac{\ln^2{(\ln{n})}}{n \ln{n}} dn = \int\limits_{\ln{3}}^\infty \frac{\ln^2{t}}{t} dt = \int\limits_{\ln{\ln{3}}}^\infty k^2 dk$$
Откуда видно, что ряд расходится. Т.к первоначальный ряд получился знакочередующийся, то проверим признак Лейбница:
\begin{enumerate}
\item $\frac{\ln^2{(\ln{(n + 1)})}}{(n + 1) \ln{(n + 1)}} < \frac{\ln^2{(\ln{n})}}{n \ln{n}}$
\item $\lim\limits_{n \to \infty} \frac{\ln^2{(\ln{n})}}{n \ln{n}} = 0$
\end{enumerate}
Признак выполняется, значит ряд $\sum\limits_{n = 3}^\infty \frac{\ln^2{(\ln{n})}}{n \ln{n}} \cos{\pi n}$ сходится. Теперь ясно, что ряд имеет условную сходимость.

\newpage
\section{Вопрос 20}
\underline{Измеримое по Жордану множество в $R^{n}$}\\

Множество $\Pi = \{(x_{1}, x_{2}, ... , x_{n}): a_{i} \leq x_{i} < b_{i}, i = 1, ... , n\}$ будем называть клеткой в $R^{n}$. Пустое множество тоже считается клеткой. Множество $A\in R^{n}$ клеточное, если оно является объединением конечного числа попарно непересекающихся клеток. \\
Мерой $m(\Pi)$ клетки называется число
\[m(\Pi) = (b_{1} - a_{1})\times...\times(b_{n} - a_{n})\]
Если непересекающиеся клетки $\Pi_{1}, ..., \Pi_{n}$ образуют разбиение клеточного множества $A$, то мерой клеточного множества $A$ назовем число
\[m(A) = \sum_{i = 1}^{n} m(\Pi_{i})\]
Множество $\Omega\subset R^{n}$ называется измеримым по Жордану, если для любого $\varepsilon > 0$ найдутся два клеточных множества $A$ и $B$ такие, что $A\subset\Omega\subset B$ и $m(B) - m(A) <\varepsilon$. (По сути, измерить множество по Жордану - значит попробовать воссоздать его с помощью прямоугольников)\\

\underline{Задача}

Найдите два клеточных множества $A$ и $B$ таких, чтобы $A\subset\Omega\subset B, m(B) - m(A) \leq 1.5$, если $\Omega = \{(x,y): 0\leq y \leq x, 0 \leq x \leq 3\}$.\\

\underline{Решение}

Чтобы понять решение, необходимо нарисовать все 3 множества на плоскости $XOY$.\\

Пусть множество $A$ - объединение клеток:\\
$\{(x, y): 0.5 \leq x \leq 1, 0 \leq y \leq 0.5\}$,\\
$\{(x, y): 1 \leq x \leq 3, 0 \leq y \leq 1\}$,\\
$\{(x, y): 1.5 \leq x \leq 2, 1 \leq y \leq 1.5\}$,\\
$\{(x, y): 2.5 \leq x \leq 3, 2 \leq y \leq 2.5\}$,\\
$\{(x, y): 2 \leq x \leq 3, 1 \leq y \leq 2\}$.

\[m(A) = 3\frac{3}{4}\]

Пусть множество $B$ - объединение клеток:
$\{(x, y): 0 \leq x \leq 1, 0 \leq y \leq 0.5\}$,\\
$\{(x, y): 0.5 \leq x \leq 1, 0.5 \leq y \leq 1\}$,\\
$\{(x, y): 1 \leq x \leq 3, 0 \leq y \leq 1\}$,\\
$\{(x, y): 1 \leq x \leq 2, 1 \leq y \leq 1.5\}$,\\
$\{(x, y): 1.5 \leq x \leq 2, 1.5 \leq y \leq 2\}$,\\
$\{(x, y): 2 \leq x \leq 3, 2 \leq y \leq 2.5\}$,\\
$\{(x, y): 2 \leq x \leq 3, 1 \leq y \leq 2\}$,\\
$\{(x, y): 2.5 \leq x \leq 3, 2.5 \leq y \leq 3\}$.

\[m(B) = 5\frac{1}{4}\]

\[m(B) - m(A) = 1.5\]
\newpage

\end{document}
